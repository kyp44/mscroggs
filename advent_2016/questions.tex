\def\advent@xvi@i{
  One of the digits of today's number was removed to leave a two digit number.
  This two digit number was added to today's number.
  The result was $619$.
}

\def\advent@xvi@ii{
  What is the maximum number of lines that can be formed by the intersection of $30$ planes?
}

\def\advent@xvi@iii{
  What is the volume of the smallest cube inside which a rectangular-based pyramid of volume $266$ will fit?
}

\def\advent@xvi@iv{
  Put the digits $1$ to $9$ (using each digit exactly once) in the boxes so that the sums are correct.
  The sums should be read left to right and top to bottom ignoring the usual order of operations.
  For example, $4 + 3 \times 2$ is $14$, not $10$.
  Today's number is the product of the digits in the red boxes.

  \grid@advent@xvi@iv{}{}{}{}{}{}{}{}{}
}

\def\advent@xvi@v{
  Today's number is the number of ways that 35 can be written as the sum of distinct numbers, with none of the numbers in the sum being divisible by $9$.

  Clarification: By "numbers", I mean (strictly) positive integers.
  The sum of the same numbers in a different order is counted as the same sum: e.g. $1 + 34$ and $34 + 1$ are not different sums.
  The trivial sum consisting of just the number $35$ counts as a sum.
}

\def\advent@xvi@vi{
  When you add up the digits of a number, the result is called the digital sum.

  How many different digital sums do the numbers from $1$ to $10^{91}$ have?
}

\newcommand\grid@advent@xvi@vii[9]{
  \begin{center}
    \begin{tikzpicture}
      \bigbox{0}{3}{#1}
      \bigbox{1}{3}{#2}
      \bigbox{2}{3}{#3}
      \bbtextr{3}{3}{multiple of $25$}

      \bigbox{0}{2}{#4}
      \bigbox{1}{2}{#5}
      \bigbox{2}{2}{#6}
      \bbtextr{3}{2}{today's number}

      \bigbox{0}{1}{#7}
      \bigbox{1}{1}{#8}
      \bigbox{2}{1}{#9}
      \bbtextr{3}{1}{all digits even}

      \bbtextb{0}{0}{multiple\\of $91$}
      \bbtextb{1}{0}{multiple\\of $7$}
      \bbtextb{2}{0}{cube\\number}
    \end{tikzpicture}
  \end{center}
}
\def\advent@xvi@vii{
  Put the digits $1$ to $9$ (using each digit once) in the boxes so that the three-digit numbers formed (reading left to right and top to bottom) have the desired properties written by their rows and columns.

  \grid@advent@xvi@vii{}{}{}{}{}{}{}{}{}
}

\def\advent@xvi@viii{
  TODO
}

\def\advent@xvi@vix{
  TODO
}

\def\advent@xvi@x{
  TODO
}

\def\advent@xvi@xi{
  TODO
}

\def\advent@xvi@xii{
  TODO
}

\def\advent@xvi@xiii{
  TODO
}

\def\advent@xvi@xiv{
  TODO
}

\def\advent@xvi@xv{
  TODO
}

\def\advent@xvi@xvi{
  TODO
}

\def\advent@xvi@xvii{
  TODO
}

\def\advent@xvi@xviii{
  TODO
}

\def\advent@xvi@xix{
  TODO
}

\def\advent@xvi@xx{
  TODO
}

\def\advent@xvi@xxi{
  TODO
}

\def\advent@xvi@xxii{
  TODO
}

\def\advent@xvi@xxiii{
  TODO
}

\def\advent@xvi@xxiv{
  TODO
}
