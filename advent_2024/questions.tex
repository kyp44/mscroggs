\def\advent@xxiv@i{
  Eve writes down five different positive integers.
  The sum of her integers is $16$. What is the product of her integers?
}

\def\advent@xxiv@ii{
  $14$ is the smallest even number that cannot be obtained by rolling two $6$-sided dice and finding the product of the numbers rolled.

  What is the smallest even number that cannot be obtained by rolling one hundred $100$-sided dice and finding the product of the numbers rolled?
}

\def\advent@xxiv@iii{
  There are $5$ ways to write $5$ as the sum of positive odd numbers:
  \begin{itemize}
    \item $1 + 1 + 1 + 1 + 1$
    \item $1 + 1 + 3$
    \item $3 + 1 + 1$
    \item $1 + 3 + 1$
    \item $5$
  \end{itemize}

  How many ways are there to write $14$ as the sum of positive odd numbers?
}

\def\advent@xxiv@iv{
  The geometric mean of a set of $n$ numbers is computed by mulitplying all the numbers together, then taking the $n$th root.
  The factors of $9$ are $1$, $3$, and $9$.
  The geometric mean of these factors is
  \gath{
    \sqrt[3]{1 \times 3 \times 9} = \sqrt[3]{27} = 3
  }
  What is the smallest number where the geometric mean of its factors is $13$?
}

\def\advent@xxiv@v{
  The sum of $11$ consecutive integers is $2024$.
  What is the smallest of the $11$ integers?
}

\def\advent@xxiv@vi{Put the digits 1 to 9 (using each digit exactly once) in the boxes so that the sums are correct. The sums should be read left to right and top to bottom ignoring the usual order of operations. For example, 4+3×2 is 14, not 10. Today's number is the product of the numbers in the red boxes.
  The number $n$ has $55$ digits.
  All of its digits are $9$.
  What is the sum of the digits of $n^3$?
}

\def\advent@xxiv@vii{
  What is the obtuse angle in degrees between the minute and hour hands of a clock at 08:22?
}

\def\advent@xxiv@viii{
  It is possible to arrange $4$ points on a plane and draw non-intersecting lines between them to form $3$ non-overlapping triangles:

  \begin{center}
    \begin{tikzpicture}
      \def\ds{3}
      \def\pa{(0: 0)}
      \def\pb{(90: \ds)}
      \def\pc{(215: 1.4*\ds)}
      \def\pd{(-30: 1.2*\ds)}

      \def\bcr{3}
      \def\scr{0.55*\bcr}
      \def\sca{34}
      \def\mcr{0.7*\bcr}
      \def\mca{142}
      \def\pr{0.1}

      % Triangles
      \draw[fill=blue,fill opacity=0.3] \pa -- \pb -- \pc -- cycle;
      \draw[fill=red,fill opacity=0.3] \pa -- \pb -- \pd -- cycle;
      \draw[fill=yellow,fill opacity=0.3] \pa -- \pd -- \pc -- cycle;

      % Points
      \fill \pa circle (\pr);
      \fill \pb circle (\pr);
      \fill \pc circle (\pr);
      \fill \pd circle (\pr);
    \end{tikzpicture}
  \end{center}

  It is not possible to make more than $3$ triangles with $4$ points.

  What is the maximum number of non-overlapping triangles that can be made by arranging $290$ points on a plane and drawing non-intersecting lines between them?
}

\def\advent@xxiv@ix{
  Put the digits $1$ to $9$ (using each digit exactly once) in the boxes so that the sums are correct.
  The sums should be read left to right and top to bottom ignoring the usual order of operations.
  For example, $4 + 3 \times 2$ is $14$, not $10$.
  Today's number is the product of the numbers in the red boxes.

  \grid@advent@xxiv@ix{}{}{}{}{}{}{}{}{}
}

\def\advent@xxiv@x{
  A number is a palindrome if it's the same when its digits are written in reverse order.

  What is the sum of all the numbers between $10$ and $100$ that are palindromes?
}

\def\advent@xxiv@xi{
  There are $6$ sets of integers between $1$ and $5$ (inclusive) that contain an odd number of numbers whose median value is $3$:

  \begin{itemize}
    \item $\braces{3}$
    \item $\braces{1,3,4}$
    \item $\braces{2,3,4}$
    \item $\braces{1,3,5}$
    \item $\braces{2,3,5}$
    \item $\braces{1,2,3,4,5}$
  \end{itemize}

  How many sets of integers between $1$ and $11$ (inclusive) are there that contain an odd number of numbers whose median value is $5$?
}

\def\advent@xxiv@xii{
  Holly picks a three-digit number.
  She then makes a two-digit number by removing one of the digits.
  The sum of her two numbers is $309$.
  What was Holly's original three-digit number?
}

\def\advent@xxiv@xiii{
  Today's number is given in this crossnumber.
  No number in the completed grid starts with $0$.

  \begin{multicols}{2}
    \crossnumstd{}{}{}{}{}{}{}{}{}

    \vfill\null
    \columnbreak

    \begin{center}
      \textbf{Across}

      \begin{tabular}{clc}
        \textbf{1} & Today's number.  & (\textbf{3}) \\
        \textbf{4} & Two times 5A.    & (\textbf{3}) \\
        \textbf{5} & A multiple of 1. & (\textbf{3})
      \end{tabular}

      \textbf{Down}

      \begin{tabular}{clc}
        \textbf{1} & Sum of digits is 15. & (\textbf{3}) \\
        \textbf{2} & Sum of digits is 19. & (\textbf{3}) \\
        \textbf{3} & Three times 5A.      & (\textbf{3})
      \end{tabular}
    \end{center}
  \end{multicols}
}

\def\advent@xxiv@xiv{
  $15^3$ is $3375$.
  The last $3$ digits of $15^3$ are $375$.

  What are the last $3$ digits of $15^{1234567890}$?
}

\def\advent@xxiv@xv{
  The number $2268$ is equal to the product of a square number (whose last digit is not $0$) and the same square number with its digits reversed: $36 \times 63$.

  What is the smallest three-digit number that is equal to the product of a square number (whose last digit is not $0$) and the same square number with its digits reversed?
}

\def\advent@xxiv@xvi{
  Put the digits $1$ to $9$ (using each digit exactly once) in the boxes so that the sums are correct.
  The sums should be read left to right and top to bottom ignoring the usual order of operations.
  For example, $4 + 3 \times 2$ is $14$, not $10$.
  Today's number is the product of the numbers in the red boxes.

  \grid@advent@xxiv@xvi{}{}{}{}{}{}{}{}{}
}

\def\advent@xxiv@xvii{
  The number $40$ has $8$ factors: $1$, $2$, $4$, $5$, $8$, $10$, $20$, and $40$.

  How many factors does the number $2^{26} \times 5 \times 7^5 \times 11^2$ have?
}

\def\advent@xxiv@xviii{
  If $k = 21$, then $28k \div (28 + k)$ is an integer.

  What is the largest integer $k$ such that $28k \div (28 + k)$ is an integer?
}

\def\advent@xxiv@xix{
  There are $9$ integers below $100$ whose digits are all non-zero and add up to $9$: $9$, $18$, $27$, $36$, $45$, $54$, $63$, $72$, and $81$.

  How many positive integers are there whose digits are all non-zero and add up to $9$?
}

\def\advent@xxiv@xx{
  $p(x)$ is a polynomial with integer coefficients such that:\

  \begin{itemize}
    \item $p(0) > 0$;
    \item if $x$ is a real number, $4x - 9 < p(x) < x^2 - 2x + 2$.
  \end{itemize}

  What is $p(23)$?
}

\newcommand\advent@xxiv@xxi@grid[5]{
  \begin{center}
    \begin{tikzpicture}[scale=2]
      \gridbox{0}{1}{#1}
      \gridbox{-1}{0}{#2}
      \gridbox{0}{0}{#3}
      \gridbox{1}{0}{#4}
      \gridbox{0}{-1}{#5}
    \end{tikzpicture}
  \end{center}
}

\def\advent@xxiv@xxi{
  Noel wants to write a different non-zero digit in each of the five boxes below so that the products of the digits of the three-digit numbers reading across and down are the same.

  \advent@xxiv@xxi@grid{}{}{}{}{}

  What is the smallest three-digit number that Noel could write in the boxes going across?
}

\def\advent@xxiv@xxii{
  Put the digits $1$ to $9$ (using each digit exactly once) in the boxes so that the sums are correct.
  The sums should be read left to right and top to bottom ignoring the usual order of operations.
  For example, $4 + 3 \times 2$ is $14$, not $10$.
  Today's number is the largest number that can be formed using the three digits in the red boxes.

  \grid@advent@xxiv@xxii{}{}{}{}{}{}{}{}{}
}

\def\advent@xxiv@xxiii{
  TODO
}

\def\advent@xxiv@xxiv{
  TODO
}

\def\card@xxiv@i{
  What is the largest number you can make by using the digits $1$ to $4$ to make two $2$-digit numbers, then mutiplying the two numbers together?
}

\def\card@xxiv@ii{
  What is the largest number you can make by using the digits $0$ to $9$ to make a $2$-digit number and an $8$-digit number, then mutiplying the two numbers together?
}

\def\card@xxiv@iii{
  The expansion of $(2x+3)^2$ is $4x^2 + 12x + 9$.
  The sum of the coefficients of $4x^2 + 12x + 9$ is $25$.
  What is the sum of the coefficients of the expansion of $(30x + 5)^2$?
}

\def\card@xxiv@iv{
  What is the sum of the coefficients of the expansion of $(2x+1)^{11}$?
}

\def\card@xxiv@v{
  What is the geometric mean of all the factors of $41306329$?
}

\def\card@xxiv@vi{
  What is the largest number for which the geometric mean of all its factors is $92$?
}

\def\card@xxiv@vii{
  What is the sum of all the factors of $7^4$?
}

\def\card@xxiv@viii{
  How many numbers between $1$ and $28988500000$ have an odd number of factors?
}

\def\card@xxiv@ix{
  Eve found the total of the $365$ consecutive integers starting at $500$ and the total of the next $365$ consecutive integers, then subtracted the smaller total from the larger total.
  What was her result?
}

\def\card@xxiv@x{
  Eve found the total of the $n$ consecutive integers starting at a number and the total of the next $n$ consecutive integers, then subtracted the smaller total from the larger total.
  Her result was $22344529$.
  What is the largest possible value of $n$ that she could have used?
}
