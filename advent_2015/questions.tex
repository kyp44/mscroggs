\def\advent@xv@i{
  What is area of the largest area rectangle which will fit in a circle of radius $10$?
}

\def\advent@xv@ii{
  In a week (Monday 12:01 to Monday 12:01 a week later), how many times will the minute hand of an analogue clock point in the same direction as the hour hand?
}

\def\advent@xv@iii{
  In August, I wrote about MENACE, a machine learning robot built from matchboxes which plays naughts and crosses.
  How many matchboxes are needed to build MENACE?
}

\newcommand\grid@advent@xv@iv[9]{
  \begin{center}
    \begin{tikzpicture}
      \gridbox{0}{6}{#1}
      \gridsym{1}{6}{-}
      \gridboxh{2}{6}{#2}
      \gridsym{3}{6}{+}
      \gridbox{4}{6}{#3}
      \gridsym{5}{6}{=}
      \gridsym{6}{6}{-4}

      \gridsym{0}{5}{+}
      \gridblank{1}{5}
      \gridsym{2}{5}{+}
      \gridblank{3}{5}
      \gridsym{4}{5}{+}

      \gridboxh{0}{4}{#4}
      \gridsym{1}{4}{-}
      \gridbox{2}{4}{#5}
      \gridsym{3}{4}{\div}
      \gridbox{4}{4}{#6}
      \gridsym{5}{4}{=}
      \gridsym{6}{4}{-1}

      \gridsym{0}{3}{-}
      \gridblank{1}{3}
      \gridsym{2}{3}{\div}
      \gridblank{3}{3}
      \gridsym{4}{3}{\times}

      \gridbox{0}{2}{#7}
      \gridsym{1}{2}{-}
      \gridbox{2}{2}{#8}
      \gridsym{3}{2}{\times}
      \gridboxh{4}{2}{#9}
      \gridsym{5}{2}{=}
      \gridsym{6}{2}{-30}

      \gridsym{0}{1}{=}
      \gridsym{2}{1}{=}
      \gridsym{4}{1}{=}

      \gridsym{0}{0}{0}
      \gridsym{2}{0}{2}
      \gridsym{4}{0}{54}
    \end{tikzpicture}
  \end{center}
}
\def\advent@xv@iv{
  Put the digits $1$ to $9$ (using each digit exactly once) in the boxes so that the sums reading across and down are correct.
  The sums should be read left to right and top to bottom ignoring the usual order of operations.
  For example, $4 + 3 \times 2$ is $14$, not $10$.

  \grid@advent@xv@iv{}{}{}{}{}{}{}{}{}
  
  The answer is the product of the digits in the red boxes.
}

\def\advent@xv@v{
  How many different triangles are there with a perimeter of $100$ and each side having an integer length?
}

\newcommand\grid@advent@xv@vi[9]{
  \begin{center}
    \begin{tikzpicture}
      \bigbox{0}{3}{#1}
      \bigbox{1}{3}{#2}
      \bigbox{2}{3}{#3}
      \bbtextr{3}{3}{multiple of $5$}

      \bigbox{0}{2}{#4}
      \bigbox{1}{2}{#5}
      \bigbox{2}{2}{#6}
      \bbtextr{3}{2}{multiple of $7$}

      \bigbox{0}{1}{#7}
      \bigbox{1}{1}{#8}
      \bigbox{2}{1}{#9}
      \bbtextr{3}{1}{cube number}

      \bbtextb{0}{0}{multiple\\of $9$}
      \bbtextb{1}{0}{multiple\\of $3$}
      \bbtextb{2}{0}{multiple\\of $4$}
    \end{tikzpicture}
  \end{center}
}
\def\advent@xv@vi{
  Put the digits $1$ to $9$ (using each digit once) in the boxes so that the three digit numbers formed (reading left to right and top to bottom) have the desired properties written by their rows and columns.

  \grid@advent@xv@vi{}{}{}{}{}{}{}{}{}

  Today's number is the multiple of $5$ formed in the first row.
}

\def\advent@xv@vii{
  In September, my puzzle appeared as \href{http://www.theguardian.com/science/2015/sep/14/can-you-solve-it-are-you-smarter-than-a-rugby-commentator}{Alex Bellos's Monday Puzzle}.
  The puzzle asked what the highest rugby score was which can only be made with one combination of kicks, tries and converted tries.
  
  What is the highest rugby score which can be made with $101$ different combinations of kicks, tries and converted tries?
}

\def\advent@xv@viii{
  What is the largest number of factors which a number less than a million has?
}

\def\advent@xv@ix{
  You start at A and are allowed to move either to the right or upwards.

    \begin{center}
    \begin{tikzpicture}
      \def\gs{1}
      % Grid
      \foreach \i in {0,...,6}{
        \foreach \j in {0,...,4}{
          \draw[thick] (\i * \gs, \j * \gs) rectangle (\i * \gs + \gs, \j * \gs + \gs);
        }
      }
      % Points
      \fill[color=red] (0, 0) circle (0.2) node[color=red,left] {A};
      \fill[color=blue] (7*\gs, 5*\gs) circle (0.2) node[color=blue,right] {B};
    \end{tikzpicture}  
  \end{center}
  
  How many different routes are there to get from A to B?
}

\def\advent@xv@x{
  This number is divisible by $2$.
  One more than this number is divisible by $3$.
  Two more than this number is divisible by $5$.
  Three more than this number is divisible by $7$.
  Four more than this number is divisible by $11$.
  Five more than this number is divisible by $13$.
}

\def\advent@xv@xi{
  This year, I was involved in starting \href{https://chalkdustmagazine.com/}{Chalkdust Magazine}.
  One of my roles for the magazine has been writing the \href{http://chalkdustmagazine.com/category/regulars/crossnumber/}{\pounds $100$ crossnumber puzzle}.

  What is the answer to $35$ across in the first issue's crossnumber?

  35A. The smallest number which is one more than triple its reverse. (3)
}

\newcommand\grid@advent@xv@xii[9]{
  \begin{center}
    \begin{tikzpicture}
      \gridbox{0}{6}{#1}
      \gridsym{1}{6}{+}
      \gridbox{2}{6}{#2}
      \gridsym{3}{6}{-}
      \gridbox{4}{6}{#3}
      \gridsym{5}{6}{=}
      \gridsym{6}{6}{-2}

      \gridsym{0}{5}{-}
      \gridblank{1}{5}
      \gridsym{2}{5}{-}
      \gridblank{3}{5}
      \gridsym{4}{5}{-}

      \gridboxh{0}{4}{#4}
      \gridsym{1}{4}{+}
      \gridboxh{2}{4}{#5}
      \gridsym{3}{4}{\div}
      \gridboxh{4}{4}{#6}
      \gridsym{5}{4}{=}
      \gridsym{6}{4}{4}

      \gridsym{0}{3}{+}
      \gridblank{1}{3}
      \gridsym{2}{3}{\div}
      \gridblank{3}{3}
      \gridsym{4}{3}{\times}

      \gridbox{0}{2}{#7}
      \gridsym{1}{2}{+}
      \gridbox{2}{2}{#8}
      \gridsym{3}{2}{\times}
      \gridbox{4}{2}{#9}
      \gridsym{5}{2}{=}
      \gridsym{6}{2}{50}

      \gridsym{0}{1}{=}
      \gridsym{2}{1}{=}
      \gridsym{4}{1}{=}

      \gridsym{0}{0}{4}
      \gridsym{2}{0}{-4}
      \gridsym{4}{0}{10}
    \end{tikzpicture}
  \end{center}
}
\def\advent@xv@xii{
  Put the digits $1$ to $9$ (using each digit exactly once) in the boxes so that the sums reading across and down are correct.
  The sums should be read left to right and top to bottom ignoring the usual order of operations. For example, $4 + 3 \times 2$ is $14$, not $10$.

  \grid@advent@xv@xii{}{}{}{}{}{}{}{}{}

  The answer is the product of the digits in the red boxes.
}

\newcommand\grid@advent@xv@xiii[9]{
  \begin{center}
    \begin{tikzpicture}
      \bigbox{0}{3}{#1}
      \bigbox{1}{3}{#2}
      \bigbox{2}{3}{#3}
      \bbtextr{3}{3}{odd}

      \bigbox{0}{2}{#4}
      \bigbox{1}{2}{#5}
      \bigbox{2}{2}{#6}
      \bbtextr{3}{2}{all digits even}

      \bigbox{0}{1}{#7}
      \bigbox{1}{1}{#8}
      \bigbox{2}{1}{#9}
      \bbtextr{3}{1}{sum}

      \bbtextb{0}{0}{even}
      \bbtextb{1}{0}{odd}
      \bbtextb{2}{0}{sum}
    \end{tikzpicture}
  \end{center}
}
\def\advent@xv@xiii{
  Put the digits $1$ to $9$ (using each digit once) in the boxes so that the three digit numbers formed (reading left to right and top to bottom) have the desired properties written by their rows and columns.

  The row marked sum is equal to the sum of the other two rows. The column marked sum is equal to the sum of the other two columns.
  \grid@advent@xv@xiii{}{}{}{}{}{}{}{}{}
  Today's number is the largest three digit number in this grid.
}

\def\advent@xv@xiv{
  What is the only palindromic three digit prime number which is also palindromic when written in binary?
}

\def\advent@xv@xv{
  If the numbers $1$ to $7$ are arranged $\mathbf{7,1,2,6,3,4,5}$ then each number is either larger than or a factor of the number before it.
  
  How many ways can the numbers $1$ to $7$ be arranged to that each number is either larger than or a factor of the number before it?
}

\def\advent@xv@xvi{
  Today's number is four thirds of the average (mean) of the answers for $13$th, $14$th, $15$th and $16$th December.
}

\def\advent@xv@xvii{
  In March, I posted the puzzle \href{https://www.mscroggs.co.uk/puzzles/105}{One Hundred Factorial}, which asked how many zeros $100!$ ends with.

  What is the smallest number, $n$, such that $n!$ ends with $50$ zeros?
}

\newcommand\grid@advent@xv@xviii[9]{
  \begin{center}
    \begin{tikzpicture}
      \bigbox{0}{3}{#1}
      \bigbox{1}{3}{#2}
      \bigbox{2}{3}{#3}
      \bbtextr{3}{3}{multiple of $9$}

      \bigbox{0}{2}{#4}
      \bigbox{1}{2}{#5}
      \bigbox{2}{2}{#6}
      \bbtextr{3}{2}{multiple of $3$}

      \bigbox{0}{1}{#7}
      \bigbox{1}{1}{#8}
      \bigbox{2}{1}{#9}
      \bbtextr{3}{1}{multiple of $5$}

      \bbtextb{0}{0}{multiple\\of $6$}
      \bbtextb{1}{0}{multiple\\of $4$}
      \bbtextb{2}{0}{cube\\number}
    \end{tikzpicture}
  \end{center}
}
\def\advent@xv@xviii{
  Put the digits $1$ to $9$ (using each digit once) in the boxes so that the three digit numbers formed (reading left to right and top to bottom) have the desired properties written by their rows and columns.

  \grid@advent@xv@xviii{}{}{}{}{}{}{}{}{}
  
  Today's number is the multiple of $6$ formed in the left hand column.
}

\def\advent@xv@xix{
  $1 = 1/2 + 1/4 + 1/8 + 1/16 + 1/16$.
  This is the sum of $5$ unit fractions (the numerators are $1$).

  In how many different ways can $1$ be written as the sum of $5$ unit fractions?
  (the same fractions in a different order are considered the same sum.)
}

\def\advent@xv@xx{
  TODO
}

\def\advent@xv@xxi{
  TODO
}

\def\advent@xv@xxii{
  TODO
}

\def\advent@xv@xxiii{
  TODO
}

\def\advent@xv@xxiv{
  TODO
}
