\def\advent@xxv@i{
  Some numbers contain a digit more than once (e.g. $313$, $111$, and $144$). Other numbers have digits that are all different (e.g. $123$, $307$, and $149$).

  How many three-digit numbers are there whose digits are all different?
}

\def\advent@xxv@ii{
  Eve writes down the numbers from $1$ to $10$ (inclusive).
  In total, she writes down $11$ digits.

  Noel writes down the number from $1$ to $100$ (inclusive).
  How many digits does he write down?
}

\def\advent@xxv@iii{
  Holly picks the number $513$, reverses it to get $315$, then adds the two together to make $828$.

  Ivy picks a three-digit number, reverses it, then adds the two together to make $968$.
  What is the smallest number that Ivy could have started with?
}

\def\advent@xxv@iv{
  Some numbers can be written as the sum of four consecutive numbers, for example: $142 = 34 + 35 + 36 + 37$.

  What is the mean of all the three-digit numbers that can be written as the sum of four consecutive numbers?
}

\def\advent@xxv@v{
  The number $36$ is equal to two times the product of its digits.

  What is the only (strictly positive) number that is equal to four times the product of its digits?
}

\def\advent@xxv@vi{
  Put the digits $1$ to $9$ (using each digit exactly once) in the boxes so that the sums are correct.
  The sums should be read left to right and top to bottom ignoring the usual order of operations.
  For example, $4 + 3 \times 2$ is $14$, not $10$.
  Today's number is the product of the numbers in the red boxes.

  \grid@advent@xxv@vi{}{}{}{}{}{}{}{}{}
}

\def\advent@xxv@vii{
  Carol organized a knockout competition in December 2024, which $6$ people entered.
  There were $2$ matches in the first round with the remaining two players given byes (so they went into the next round without playing a match).
  The second round was made up of two semi-finals, then one final match was played to decide the winner.
  In total $5$ matches were played.

  This year, Carol is organizing the competition again, but it has become a lot more popular: $355$ people have entered.
  While planning the tournament, she can decide which rounds to give people byes in.
  What is the smallest number of matches that could be included in the tournament?
}

\def\advent@xxv@viii{
  Angel wrote out a multiplication square for the numbers from $1$ to $3$ (the table has the numbers $1$ to $3$ in the top row and left column, then every other entry is equal to the number at the top of its column multiplied by the number at the left of its row):
  \begin{center}
    \begin{tabular}{|c|c|c|}
      \hline
      1 & 2 & 3 \\
      \hline
      2 & 4 & 6 \\
      \hline
      3 & 6 & 9 \\
      \hline
    \end{tabular}
  \end{center}

  The sum of the numbers in the bottom row is $18$. The sum of all the numbers in the table is $36$.

  Angel then wrote out another multiplication square with the numbers from $1$ to $n$.
  The sum of all the numbers in the new table is $2025$.
  What is the sum of the numbers in the bottom row of the new table?
}

\definecolor{ixhigh}{rgb}{0.18, 0.639, 0.812}
\definecolor{ixline}{rgb}{0.965, 0.655, 0.435}
\newcommand\ixbox[2]{\draw[thick] (#1,#2) rectangle (#1+1,#2+1) node[pos=.5] {};}
\newcommand\ixboxh[2]{\draw[thick,fill=ixhigh] (#1,#2) rectangle (#1+1,#2+1) node[pos=.5] {};}
\def\advent@xxv@ix{
  In a $3$ by $5$ grid of squares, if a line is drawn from the bottom left corner to the top right corner, it will pass through $7$ squares:
  \begin{center}
    \begin{tikzpicture}[scale=1.7]
      \ixboxh{0}{0}
      \ixboxh{1}{0}
      \ixbox{2}{0}
      \ixbox{3}{0}
      \ixbox{4}{0}
      \ixbox{0}{1}
      \ixboxh{1}{1}
      \ixboxh{2}{1}
      \ixboxh{3}{1}
      \ixbox{4}{1}
      \ixbox{0}{2}
      \ixbox{1}{2}
      \ixbox{2}{2}
      \ixboxh{3}{2}
      \ixboxh{4}{2}
      \draw[draw=ixline, very thick] (0,0) -- (5,3);
    \end{tikzpicture}
  \end{center}

  In a $251$ by $272$ grid of squares, how many squares will a line drawn from the bottom left corner to the top right corner pass through?
}

\def\advent@xxv@x{
  $2025$ is the smallest number with exactly $15$ odd factors.

  What is the smallest number with exactly $16$ odd factors?
}

\def\advent@xxv@xi{
  Holly added up $3$ consecutive numbers starting at $10$, then added up the next $3$ consective numbers, then found the difference between her two totals:
  \gaths{
    10 + 11 + 12 = 33 \\
    13 + 14 + 15 = 42 \\
    42 - 33 = 9
  }
  Ivy added up $n$ consecutive numbers starting at $m$, then added up the next n consecutive numbers, then found the difference between her two totals.
  The difference was $203401$.
  What is the largest possible value of n that Ivy could have used?
}

\def\advent@xxv@xii{
  Mary uses the digits $1$, $2$, $3$, $4$, $5$, $6$ and $7$ to make two three-digit numbers and a one-digit number (using each digit exactly once).
  The sum of her three numbers is $1000$.

  What is the smallest that the larger of her two three-digit numbers could be?
}

% This cheats a bit because we are annoyingly limited to 9 arguments
\newcommand\grid@advent@xxv@xiii[8]{
  \begin{center}
    \begin{tikzpicture}[scale=2]
      % Grid
      \gridbox{0}{2}{#1}
      \gridbox{1}{2}{#2}
      \gridbox{2}{2}{#3}
      \gridbox{0}{1}{#4}
      \gridbox{1}{1}{#5}
      \gridbox{2}{1}{#6}
      \gridbox{1}{0}{#7}
      \gridbox{2}{0}{#8}

      % Blacked out boxes
      \gridboxb{0}{0}{}

      % Across / Down labels
      \cwlab{0}{2}{1}
      \cwlab{1}{2}{2}
      \cwlab{2}{2}{3}
      \cwlab{0}{1}{4}
      \cwlab{1}{0}{5}

    \end{tikzpicture}
  \end{center}
}
\def\advent@xxv@xiii{
  Today's number is given in this crossnumber.
  No number in the completed grid starts with 0.

  \setlength{\columnsep}{-2cm}
  \begin{multicols}{2}
    \grid@advent@xxv@xiii{}{}{}{}{}{}{}{}

    \vfill\null
    \columnbreak

    \begin{center}
      \textbf{Across}

      \begin{tabular}{clc}
        \textbf{1} & Today's number.                              & (\textbf{3}) \\
        \textbf{4} & Two times today's number.                    & (\textbf{3}) \\
        \textbf{5} & The product of the digits of today's number. & (\textbf{2})
      \end{tabular}

      \textbf{Down}

      \begin{tabular}{clc}
        \textbf{1} & The sum of the digits of today's number.    & (\textbf{2}) \\
        \textbf{2} & Two more than an anagram of today's number. & (\textbf{3}) \\
        \textbf{3} & A multiple of $101$.                        & (\textbf{3})
      \end{tabular}
    \end{center}
  \end{multicols}
}

\def\advent@xxv@xiv{
  There are five ways to make a list of four As and Bs that don't contain an odd number of consecutive As:
  \begin{itemize}
    \item B,B,B,B
    \item A,A,B,B
    \item B,A,A,B
    \item B,B,A,A
    \item A,A,A,A
  \end{itemize}
  How many ways are there to make a list of eleven As and Bs that don't contain an odd number of consecutive As?
}

\def\advent@xxv@xv{
  The odd factors of $2025$ are $1$, $3$, $5$, $9$, $15$, $25$, $27$, $45$, $75$, $81$, $135$, $225$, $405$, $675$ and $2025$.
  There are $15$ of these factors and $15$ is itself an odd factor of $2025$.

  What is the smallest three-digit number whose number of odd factors is itself an odd factor of the number?
}

\def\advent@xxv@xvi{
  Put the digits $1$ to $9$ (using each digit exactly once) in the boxes so that the sums are correct.
  The sums should be read left to right and top to bottom ignoring the usual order of operations.
  For example, $4 + 3 \times 2$ is $14$, not $10$.
  Today's number is the product of the numbers in the red boxes.

  \grid@advent@xxv@xvi{}{}{}{}{}{}{}{}{}
}

\newcommand\xviibox[3]{\draw (#1,#2) rectangle (#1+1,#2+1) node[pos=.5] {#3};}
\def\advent@xxv@xvii{
  A sequence of zeros and ones can be reduced by writing a $0$ or $1$ under each pair of numbers: $1$ is written if the numbers are the same, $0$ is written if they are not.
  This process can be repeated until there is a single number.
  For example, if we start with the sequence $1$, $1$, $1$, $0$, $1$ (of length $5$), we get:

  \begin{center}
    \begin{tikzpicture}[scale=1.2]
      \xviibox{0}{0}{1}

      \xviibox{-0.5}{1}{0}
      \xviibox{0.5}{1}{0}

      \xviibox{-1}{2}{1}
      \xviibox{0}{2}{0}
      \xviibox{1}{2}{1}

      \xviibox{-1.5}{3}{1}
      \xviibox{-0.5}{3}{1}
      \xviibox{0.5}{3}{0}
      \xviibox{1.5}{3}{0}

      \xviibox{-2}{4}{1}
      \xviibox{-1}{4}{1}
      \xviibox{0}{4}{1}
      \xviibox{1}{4}{0}
      \xviibox{2}{4}{1}
    \end{tikzpicture}
  \end{center}

  The final digit is a $1$.

  How many sequences of zeros and ones of length $10$ are there that when reduced lead to the final digit being a $1$?
}

\def\advent@xxv@xviii{
  There are $5$ different ways to make a set of numbers between $1$ and $5$ such that the smallest number in the set is equal to the number of numbers in the set.
  These $5$ sets are: $\braces{1}$, $\braces{2, 3}$, $\braces{2, 4}$, $\braces{2, 5}$ and $\braces{3, 4, 5}$.

  How many ways are there to make a set of numbers between $1$ and $14$ such that the smallest number in the set is equal to the number of numbers in the set?
}

\def\advent@xxv@xix{
  Eve uses the digits $1$, $2$  , $3$, $4$, $5$, $6$, $7$, $8$ and $9$ to write five square numbers (using each digit exactly once).
  What is largest square number that she made?
}

\def\advent@xxv@xx{
  A number is called a perfect power if it is equal to $n^k$ for some integer $n$ and some integer $k > 1$.
  $2025$ is a perfect power ($45^2$) and $23$ more than $2025$ is also a perfect power ($2^{11}$).

  What is the only three-digit perfect power that is $29$ less than another perfect power?
}

\def\advent@xxv@xxi{
  TODO
}

\def\advent@xxv@xxii{
  TODO
}

\def\advent@xxv@xxiii{
  TODO
}

\def\advent@xxv@xxiv{
  TODO
}
