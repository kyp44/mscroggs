% Monospace code
\def\code#1{\texttt{#1}}

% Greek letters
\def\a{\alpha}
\def\b{\beta}
\def\g{\gamma}
\def\d{\delta}
\def\D{\Delta}

% Some common sets
\def\es{\varnothing}
\def\ints{{\mathbb{Z}}}

% Commands that make life easier
\newcommand\gath[1]{\begin{gather} #1 \end{gather}}
\newcommand\gaths[1]{\begin{gather*} #1 \end{gather*}}
\newcommand\ali[1]{\begin{align} #1 \end{align}}
\newcommand\parens[1]{\left( #1 \right)}
\newcommand\squares[1]{\left[ #1 \right]}
\newcommand\braces[1]{\left\{ #1 \right\}}
\newcommand\angles[1]{\left\langle #1 \right\rangle}
\newcommand\deriv[2]{\frac{d #1}{d #2}}
\newcommand\abs[1]{\left| #1 \right|}
\newcommand\floor[1]{\left\lfloor #1 \right\rfloor}
\newcommand\ceil[1]{\left\lceil #1 \right\rceil}
\DeclareMathOperator{\lcm}{lcm}
\def\non{\nonumber \\}
\newcommand\unit[1]{~\mathrm{#1}}

% Set stuff
\def\ss{\subseteq}

% Multiline equation space
\def\mlesp{\hspace{1.2cm}}

% For grid diagrams
\newcommand\gridbox[3]{\draw (#1,#2) rectangle (#1+1,#2+1) node[pos=.5] {#3};}
\newcommand\gridboxh[3]{\draw[fill=red!20] (#1,#2) rectangle (#1+1,#2+1) node[pos=.5] {#3};}
\newcommand\gridboxb[3]{\draw[fill=black] (#1,#2) rectangle (#1+1,#2+1) node[pos=.5] {#3};}
\newcommand\gridsym[3]{\node at (#1+0.5,#2+0.5) {$#3$};}
\newcommand\gridblank[2]{\filldraw[draw=gray, color=gray] (#1,#2) rectangle (#1+1,#2+1);}
\newcommand\gridcirc[2]{\draw (#1 + 0.5,#2 + 0.5) circle (0.25);}
\newcommand\cwlab[3]{
  \def\dd{0.15}
  \draw (#1 + \dd - 0.03, #2 + 1 - \dd) node {\scriptsize #3};
}

\def\bbw{3.5}
\def\bbh{2}
\newcommand\bigbox[3]{\draw (#1*\bbw,#2*\bbh) rectangle (#1*\bbw+\bbw,#2*\bbh+\bbh) node[pos=.5] {#3};}
\newcommand\bbtextr[3]{\node[right] at (#1*\bbw,#2*\bbh+0.5*\bbh) {#3};}
\newcommand\bbtextb[3]{\node[align=center] at (#1*\bbw+0.5*\bbw,#2*\bbh+0.5*\bbh) {#3};}

% Box puzzle stock answer
\newcommand\boxans[1]{
  Logic was used to deduce the solution:

  #1

  This was verified using Python as well as shown to be unique with a brute force approach.
}

% Standard crossnumber grid
\newcommand\crossnumstd[9]{
  \begin{center}
    \begin{tikzpicture}[scale=2]
      \gridbox{0}{2}{#1}
      \gridbox{1}{2}{#2}
      \gridbox{2}{2}{#3}
      \gridbox{0}{1}{#4}
      \gridbox{1}{1}{#5}
      \gridbox{2}{1}{#6}
      \gridbox{0}{0}{#7}
      \gridbox{1}{0}{#8}
      \gridbox{2}{0}{#9}

      % Labels
      \cwlab{0}{2}{1}
      \cwlab{1}{2}{2}
      \cwlab{2}{2}{3}
      \cwlab{0}{1}{4}
      \cwlab{0}{0}{5}
    \end{tikzpicture}
  \end{center}
}

% Multiple numbers
\newcommand\mn[1]{$#1$'s}

% Commands for problems
\newcommand\problem[4]{
\section*{#1}

\textbf{Question:} #3

\textbf{Answer:} #2

\textbf{Explanation:} #4
}
\newcommand\aproblem[4]{\problem{Dec #1}{#2}{#3}{#4}}
\newcommand\cproblem[4]{\problem{Problem #1}{#2}{#3}{#4}}

\newcommand\xref@advent[2]{#1 Advent, Dec~#2 problem}
\newcommand\xref@card[2]{#1 Christmas Card, Problem #2}
