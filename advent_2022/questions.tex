\def\advent@xxii@i{
  One of the vertices of a rectangle is at the point $(9, 0)$.
  The $x$-axis and $y$-axis are both lines of symmetry of the rectangle.

  What is the area of the rectangle?
}

\def\advent@xxii@ii{
  What is the smallest number that is a multiple of $1$, $2$, $3$, $4$, $5$, $6$, $7$, and $8$?
}

\def\advent@xxii@iii{
  Write the numbers $1$ to $81$ in a grid like this:
  \gath{
    \begin{array}{ccccc}
      1      & 2      & 3      & \cdots & 9      \\
      10     & 11     & 12     & \cdots & 18     \\
      19     & 20     & 21     & \cdots & 27     \\
      \vdots & \vdots & \vdots & \ddots & \vdots \\
      73     & 74     & 75     & \cdots & 81
    \end{array}
  }
  Pick $9$ numbers so that you have exactly one number in each row and one number in each column, and find their sum. What is the largest value you can get?
}

\def\advent@xxii@iv{
  The last three digits of $5^5$ are $125$.
  What are the last three digits of $5^{2,022,000,000}$?
}

\def\advent@xxii@v{
  Put the digits $1$ to $9$ (using each digit exactly once) in the boxes so that the sums are correct.
  The sums should be read left to right and top to bottom ignoring the usual order of operations.
  For example, $4 + 3 \times 2$ is $14$, not $10$.
  Today's number is the product of the numbers in the red boxes.

  \grid@advent@xxii@v{}{}{}{}{}{}{}{}{}
}

\def\advent@xxii@vi{
  There are $21$ three-digit integers whose digits are all non-zero and whose digits add up to $8$.

  How many positive integers are there whose digits are all non-zero and whose digits add up to $8$?
}

\def\advent@xxii@vii{
  What is the area of the largest triangle that fits inside a regular hexagon with area $952$?
}

\def\advent@xxii@viii{
  The equation $x^5 - 7x^4 - 27x^3 + 175x^2 + 218x = 840$ has five real solutions.
  What is the product of all these solutions?
}

\def\advent@xxii@ix{
  Put the digits $1$ to $9$ (using each digit exactly once) in the boxes so that the sums are correct.
  The sums should be read left to right and top to bottom ignoring the usual order of operations.
  For example, $4 + 3 \times 2$ is $14$, not $10$.
  Today's number is the largest number you can make with the digits in the red boxes.

  \grid@advent@xxii@ix{}{}{}{}{}{}{}{}{}
}

\def\advent@xxii@x{
  TODO
}

\def\advent@xxii@xi{
  TODO
}

\def\advent@xxii@xii{
  TODO
}

\def\advent@xxii@xiii{
  TODO
}

\def\advent@xxii@xiv{
  TODO
}

\def\advent@xxii@xv{
  TODO
}

\def\advent@xxii@xvi{
  TODO
}

\def\advent@xxii@xvii{
  TODO
}

\def\advent@xxii@xviii{
  TODO
}

\def\advent@xxii@xix{
  TODO
}

\def\advent@xxii@xx{
  TODO
}

\def\advent@xxii@xxi{
  TODO
}

\def\advent@xxii@xxii{
  TODO
}

\def\advent@xxii@xxiii{
  TODO
}

\def\advent@xxii@xxiv{
  TODO
}

\def\card@xxii@i{
  What is the only prime number that is both two more than a prime number and two less than a prime number?
}

\def\card@xxii@ii{
  Holly adds up the first $7$ odd numbers.
  What total does she get?
}

\def\card@xxii@iii{
  Holly next adds up the first $n$ odd numbers to get $1089$ a total of 1089. What is $n$?
}

\def\card@xxii@iv{
  Ivy starts with $0$ then adds or subtracts some multiples of $4$ or $7$.
  What is the smallest positive integer that she could have ended with?
}

\def\card@xxii@v{
  Ivy again starts with $0$, but this time she adds or subtracts some multiples of $240$ or $400$.
  What is the smallest positive integer that she could have ended with?
}

\def\card@xxii@vi{
  How many $4$-digit integers are there whose digits are all non-zero and whose digits add up to $7$?
}

\def\card@xxii@vii{
  How many positive integers are there whose digits are all non-zero and whose digits add up to $7$?
}

\def\card@xxii@viii{
  Eve wrote down a four-digit number.
  Eve then removed one of the digits of her number to make a three-digit number.
  The sum of her two numbers is $3119$.
  What was her four-digit number?
}

\def\card@xxii@ix{
  Eve wrote down a five-digit number.
  Eve then removed one of the digits of her number to make a four-digit number.
  The sum of her two numbers is $96158$.
  What is the largest number that her five-digit number could have been?
}

\def\card@xxii@x{
  Noel drew $12$ points on the circumference of a circle, then drew a straight line connecting every pair of points.
  How many lines did he draw?
}

\def\card@xxii@xi{
  Noel drew some points on the circumference of a circle, then drew a straight line connecting every pair of points.
  He drew $2926$ lines.
  How many points did he draw?
}
