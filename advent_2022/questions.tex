\def\advent@xxii@i{
  One of the vertices of a rectangle is at the point $(9, 0)$.
  The $x$-axis and $y$-axis are both lines of symmetry of the rectangle.

  What is the area of the rectangle?
}

\def\advent@xxii@ii{
  What is the smallest number that is a multiple of $1$, $2$, $3$, $4$, $5$, $6$, $7$, and $8$?
}

\def\advent@xxii@iii{
  Write the numbers $1$ to $81$ in a grid like this:
  \gath{
    \begin{array}{ccccc}
      1      & 2      & 3      & \cdots & 9      \\
      10     & 11     & 12     & \cdots & 18     \\
      19     & 20     & 21     & \cdots & 27     \\
      \vdots & \vdots & \vdots & \ddots & \vdots \\
      73     & 74     & 75     & \cdots & 81
    \end{array}
  }
  Pick $9$ numbers so that you have exactly one number in each row and one number in each column, and find their sum. What is the largest value you can get?
}

\def\advent@xxii@iv{
  The last three digits of $5^5$ are $125$.
  What are the last three digits of $5^{2,022,000,000}$?
}

\def\advent@xxii@v{
  Put the digits $1$ to $9$ (using each digit exactly once) in the boxes so that the sums are correct.
  The sums should be read left to right and top to bottom ignoring the usual order of operations.
  For example, $4 + 3 \times 2$ is $14$, not $10$.
  Today's number is the product of the numbers in the red boxes.

  \grid@advent@xxii@v{}{}{}{}{}{}{}{}{}
}

\def\advent@xxii@vi{
  There are $21$ three-digit integers whose digits are all non-zero and whose digits add up to $8$.

  How many positive integers are there whose digits are all non-zero and whose digits add up to $8$?
}

\def\advent@xxii@vii{
  What is the area of the largest triangle that fits inside a regular hexagon with area $952$?
}

\def\advent@xxii@viii{
  The equation $x^5 - 7x^4 - 27x^3 + 175x^2 + 218x = 840$ has five real solutions.
  What is the product of all these solutions?
}

\def\advent@xxii@ix{
  Put the digits $1$ to $9$ (using each digit exactly once) in the boxes so that the sums are correct.
  The sums should be read left to right and top to bottom ignoring the usual order of operations.
  For example, $4 + 3 \times 2$ is $14$, not $10$.
  Today's number is the largest number you can make with the digits in the red boxes.

  \grid@advent@xxii@ix{}{}{}{}{}{}{}{}{}
}

\def\advent@xxii@x{
  A line is tangent to a curve if the line touches the curve at exactly one point.
  The line $y=-160,000$ is tangent to the parabola $y=x^2-ax$.
  What is $a$?
}

\def\advent@xxii@xi{
  There are five $3$-digit numbers whose digits are all either $1$ or $2$ and who do not contain two \mn{2} in a row: $111$, $112$, $121$, $211$, and $212$.

  How many $14$-digit numbers are there whose digits are all either $1$ or $2$ and who do not contain two \mn{2} in a row?
}

\def\advent@xxii@xii{
  The determinant of the $2$ by $2$ matrix $\begin{pmatrix} a & b \\ c & d \end{pmatrix}$ is $ad - bc$.

  If a $2$ by $2$ matrix's entries are all in the set $\braces{1, 2, 3}$, the largest possible determinant of this matrix is $8$.

  What is the largest possible determinant of a $2$ by $2$ matrix whose entries are all in the set $\braces{1, 2, 3, ..., 12}$?
}

\newcommand\grid@advent@xxii@xiii[9]{
  \begin{center}
    \begin{tikzpicture}[scale=2]
      \gridbox{0}{2}{#1}
      \gridbox{1}{2}{#2}
      \gridbox{2}{2}{#3}
      \gridbox{0}{1}{#4}
      \gridbox{1}{1}{#5}
      \gridbox{2}{1}{#6}
      \gridbox{0}{0}{#7}
      \gridbox{1}{0}{#8}
      \gridbox{2}{0}{#9}

      % Labels
      \cwlab{0}{2}{1}
      \cwlab{1}{2}{2}
      \cwlab{2}{2}{3}
      \cwlab{0}{1}{4}
      \cwlab{0}{0}{5}
    \end{tikzpicture}
  \end{center}
}
\def\advent@xxii@xiii{
  Today's number is given in this crossnumber.
  The across clues are given as normal, but the down clues are given in a random order: you must work out which clue goes with each down entry and solve the crossnumber to find today's number.
  No number in the completed grid starts with $0$.
  \begin{multicols}{2}
    \grid@advent@xxii@xiii{}{}{}{}{}{}{}{}{}

    \vfill\null
    \columnbreak

    \begin{center}
      \textbf{Across}

      \begin{tabular}{clc}
        \textbf{1} & A cube number.    & (\textbf{3}) \\
        \textbf{4} & A square number.  & (\textbf{3}) \\
        \textbf{5} & A multiple of 13. & (\textbf{3})
      \end{tabular}

      \textbf{Down} \\
      (in a random order)
      \begin{itemize}[leftmargin=2.6cm, itemsep=1pt, parsep=1pt]
        \item Two times 1A.
        \item Today's number.
        \item A square number.
      \end{itemize}
    \end{center}
  \end{multicols}
}

\def\pr{3}
\def\pd{1.6180339887498951*\pr}
\def\rd{1.902113032590307*\pd}
\newcommand\pentagon@up[1]{\draw[line width=1, color=blue, shift={#1}] (-126: \pr) -- (-54: \pr) -- (18: \pr) -- (90: \pr) -- (162: \pr) -- cycle;}
\newcommand\pentagon@down[1]{\draw[line width=1, color=blue, shift={#1}] (54: \pr) -- (126: \pr) -- (198: \pr) -- (270: \pr) -- (342: \pr) -- cycle;}
\newcommand\downpos[1]{({#1 * \rd + \pd * cos(-18)}, {\pd * sin(-18)})}
\def\advent@xxii@xiv{
  Holly draws a line of connected regular pentagons like this:

  \begin{center}
    \begin{tikzpicture}
      \foreach \n in {0,...,2}
        {
          \pentagon@up{(\n * \rd, 0)}
        }
      \foreach \n in {0,...,1}
        {
          \pentagon@down{\downpos{\n}}
        }
      \draw[line width=1, color=blue, shift={\downpos{2}}, dashed] (54: \pr) -- (126: \pr);
      \draw[line width=1, color=blue, shift={\downpos{2}}, dashed] (198: \pr) -- (270: \pr);
    \end{tikzpicture}
  \end{center}

  She continues the pattern until she has drawn $204$ pentagons.
  The perimeter of each pentagon is $5$.
  What is the perimeter of her line of pentagons?
}

\def\advent@xxii@xv{
  There are $3$ even numbers between $3$ and $9$.

  What is the only odd number $n$ such that there are $n$ even numbers between $n$ and $729$?
}

\def\advent@xxii@xvi{
  Noel writes the integers from $1$ to $1000$ in a large triangle like this:
  \begin{center}
    \begin{tabular}{ccccccc}
         &    &    & 1  &          &   & \\
         &    & 2  & 3  & 4        &   & \\
         & 5  & 6  & 7  & 8        & 9 & \\
      10 & 11 & 12 & 13 & $\ldots$ &   &
    \end{tabular}
  \end{center}
  The rightmost number in the row containing the number $6$ is $9$.
  What is the rightmost number in the row containing the number $300$?
}

\def\advent@xxii@xvii{
  Put the digits $1$ to $9$ (using each digit exactly once) in the boxes so that the sums are correct.
  The sums should be read left to right and top to bottom ignoring the usual order of operations.
  For example, $4 + 3 \times 2$ is $14$, not $10$.
  Today's number is the product of the numbers in the red boxes.

  \grid@advent@xxii@xvii{}{}{}{}{}{}{}{}{}
}

\def\advent@xxii@xviii{
  Noel writes the integers from $1$ to $1000$ in a large triangle like this

  \begin{center}
    \begin{tabular}{ccccccc}
         &    &    & 1  &          &   & \\
         &    & 2  & 3  & 4        &   & \\
         & 5  & 6  & 7  & 8        & 9 & \\
      10 & 11 & 12 & 13 & $\ldots$ &   &
    \end{tabular}
  \end{center}

  The number $12$ is directly below the number $6$.
  Which number is directly below the number $133$?
}

\def\advent@xxii@xix{
  $120$ is the smallest number with exactly $16$ factors (including $1$ and $120$ itself).

  What is the second-smallest number with exactly $16$ factors (including $1$ and the number itself)?
}

\def\advent@xxii@xx{
  TODO
}

\def\advent@xxii@xxi{
  TODO
}

\def\advent@xxii@xxii{
  TODO
}

\def\advent@xxii@xxiii{
  TODO
}

\def\advent@xxii@xxiv{
  TODO
}

\def\card@xxii@i{
  What is the only prime number that is both two more than a prime number and two less than a prime number?
}

\def\card@xxii@ii{
  Holly adds up the first $7$ odd numbers.
  What total does she get?
}

\def\card@xxii@iii{
  Holly next adds up the first $n$ odd numbers to get $1089$ a total of 1089. What is $n$?
}

\def\card@xxii@iv{
  Ivy starts with $0$ then adds or subtracts some multiples of $4$ or $7$.
  What is the smallest positive integer that she could have ended with?
}

\def\card@xxii@v{
  Ivy again starts with $0$, but this time she adds or subtracts some multiples of $240$ or $400$.
  What is the smallest positive integer that she could have ended with?
}

\def\card@xxii@vi{
  How many $4$-digit integers are there whose digits are all non-zero and whose digits add up to $7$?
}

\def\card@xxii@vii{
  How many positive integers are there whose digits are all non-zero and whose digits add up to $7$?
}

\def\card@xxii@viii{
  Eve wrote down a four-digit number.
  Eve then removed one of the digits of her number to make a three-digit number.
  The sum of her two numbers is $3119$.
  What was her four-digit number?
}

\def\card@xxii@ix{
  Eve wrote down a five-digit number.
  Eve then removed one of the digits of her number to make a four-digit number.
  The sum of her two numbers is $96158$.
  What is the largest number that her five-digit number could have been?
}

\def\card@xxii@x{
  Noel drew $12$ points on the circumference of a circle, then drew a straight line connecting every pair of points.
  How many lines did he draw?
}

\def\card@xxii@xi{
  Noel drew some points on the circumference of a circle, then drew a straight line connecting every pair of points.
  He drew $2926$ lines.
  How many points did he draw?
}
