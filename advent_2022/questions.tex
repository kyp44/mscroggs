\def\advent@xxii@i{
  One of the vertices of a rectangle is at the point $(9, 0)$.
  The $x$-axis and $y$-axis are both lines of symmetry of the rectangle.

  What is the area of the rectangle?
}

\def\advent@xxii@ii{
  What is the smallest number that is a multiple of $1$, $2$, $3$, $4$, $5$, $6$, $7$, and $8$?
}

\def\advent@xxii@iii{
  Write the numbers $1$ to $81$ in a grid like this:
  \gath{
    \begin{array}{ccccc}
      1      & 2      & 3      & \cdots & 9      \\
      10     & 11     & 12     & \cdots & 18     \\
      19     & 20     & 21     & \cdots & 27     \\
      \vdots & \vdots & \vdots & \ddots & \vdots \\
      73     & 74     & 75     & \cdots & 81
    \end{array}
  }
  Pick $9$ numbers so that you have exactly one number in each row and one number in each column, and find their sum. What is the largest value you can get?
}

\def\advent@xxii@iv{
  The last three digits of $5^5$ are $125$.
  What are the last three digits of $5^{2,022,000,000}$?
}

\def\advent@xxii@v{
  Put the digits $1$ to $9$ (using each digit exactly once) in the boxes so that the sums are correct.
  The sums should be read left to right and top to bottom ignoring the usual order of operations.
  For example, $4 + 3 \times 2$ is $14$, not $10$.
  Today's number is the product of the numbers in the red boxes.

  \grid@advent@xxii@v{}{}{}{}{}{}{}{}{}
}

\def\advent@xxii@vi{
  TODO
}

\def\advent@xxii@vii{
  TODO
}

\def\advent@xxii@viii{
  TODO
}

\def\advent@xxii@ix{
  TODO
}

\def\advent@xxii@x{
  TODO
}

\def\advent@xxii@xi{
  TODO
}

\def\advent@xxii@xii{
  TODO
}

\def\advent@xxii@xiii{
  TODO
}

\def\advent@xxii@xiv{
  TODO
}

\def\advent@xxii@xv{
  TODO
}

\def\advent@xxii@xvi{
  TODO
}

\def\advent@xxii@xvii{
  TODO
}

\def\advent@xxii@xviii{
  TODO
}

\def\advent@xxii@xix{
  TODO
}

\def\advent@xxii@xx{
  TODO
}

\def\advent@xxii@xxi{
  TODO
}

\def\advent@xxii@xxii{
  TODO
}

\def\advent@xxii@xxiii{
  TODO
}

\def\advent@xxii@xxiv{
  TODO
}