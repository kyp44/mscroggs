\def\advent@xxi@i{
  The geometric mean of a set of $n$ numbers can be computed by multiplying together all the numbers then computing the $n$th root of the result.

  The factors of $4$ are $1$, $2$ and $4$. The geometric mean of these is 2.

  The factors of $6$ are $1$, $2$, $3$, and $6$. The geometric mean of these is $\sqrt{6}$.

  The geometric mean of all the factors of today's number is $22$.
}

\def\advent@xxi@ii{
  The number $7n$ has $37$ factors (including $1$ and the number itself).
  How many factors does $8n$ have?
}

\def\advent@xxi@iii{
  If you write out the numbers from $1$ to $1000$ (inclusive), how many times will you write the digit $0$?
}

\def\advent@xxi@iv{
  Put the digits $1$ to $9$ (using each digit exactly once) in the boxes so that the sums are correct.
  The sums should be read left to right and top to bottom ignoring the usual order of operations.
  For example, $4 + 3 \times 2$ is $14$, not $10$.
  Today's number is the product of the numbers in the red boxes.

  \grid@advent@xxi@iv{}{}{}{}{}{}{}{}{}
}

\def\advent@xxi@v{
  How many different isosceles triangles are there whose perimeter is $50$ units, and whose area is an integer number of units squared?

  (Two triangles that are rotations, reflections and translations of each other are counted as the same triangle. Triangles with an area of 0 should not be counted.)
}

\def\advent@xxi@vi{
  When $12345$ is divided by today's number, the remainder is $205$.
  When $6789$ is divided by today's number, the remainder is $112$.
}

\newcommand\dec@ai{0.30901699437494745}
\newcommand\dec@aii{0.8090169943749475}
\newcommand\dec@bi{0.5877852522924731}
\newcommand\dec@bii{0.9510565162951535}
\newcommand\decagon[5]{
  \def\ai{\dec@ai*#3+#1}
  \def\aii{\dec@aii*#3+#1}
  \def\bi{\dec@bi*#3+#2}
  \def\bii{\dec@bii*#3+#2}
  \draw (#3+#1, #2) -- (\aii, \bi) -- (\ai, \bii) -- (-\ai, \bii) -- (-\aii, \bi) -- (-#3+#1, #2) -- (-\aii, -\bi) -- (-\ai, -\bii) -- (\ai, -\bii) -- (\aii, -\bi) -- cycle;
  \fill[fill=red] (-\ai, -\bii) -- (#4*#3+#1, #5*#3+#2) -- (\ai, -\bii) -- cycle;
}
\def\advent@xxi@vii{
  The picture below shows eight regular decagons.
  In each decagon, a red triangle has been drawn with vertices at three of the vertices of the decagon.

  \begin{center}
    \begin{tikzpicture}
      \def\dr{2}
      \def\spc{2.3*\dr}

      \decagon{0*\spc}{\spc}{\dr}{-\dec@aii}{-\dec@bi}
      \decagon{1*\spc}{\spc}{\dr}{-1}{0}
      \decagon{2*\spc}{\spc}{\dr}{-\dec@aii}{\dec@bi}
      \decagon{3*\spc}{\spc}{\dr}{-\dec@ai}{\dec@bii}

      \decagon{0*\spc}{0}{\dr}{\dec@ai}{\dec@bii}
      \decagon{1*\spc}{0}{\dr}{\dec@aii}{\dec@bi}
      \decagon{2*\spc}{0}{\dr}{1}{0}
      \decagon{3*\spc}{0}{\dr}{\dec@aii}{-\dec@bi}
    \end{tikzpicture}
  \end{center}

  The area of each decagon is $240$.
  What is the total area of all the red triangles?
}

\def\advent@xxi@viii{
  The sum of three integers is $51$.
  The product of the same three integers is $836$. What is the product of largest integer and the second-largest integer?
}

\def\advent@xxi@ix{
  Eve writes down a sequence of consecutive positive integers (she writes more than one number).
  The sum of the numbers Eve has written down is $844$.
  Today's number is the smallest integer that Eve has written down.
}

\def\advent@xxi@x{
  Put the digits $1$ to $9$ (using each digit exactly once) in the boxes so that the sums are correct.
  Today's number is the largest number you can make using the digits in the red boxes.

  \grid@advent@xxi@x{}{}{}{}{}{}{}{}{}
}

\def\advent@xxi@xi{
  The integers are written in a triangle as shown below:
  \begin{center}
    \begin{tabular}{ccccccc}
         &    &    & 1    &    &    &    \\
         &    & 2  & 3    & 4  &    &    \\
         & 5  & 6  & 7    & 8  & 9  &    \\
      10 & 11 & 12 & 13   & 14 & 15 & 16 \\
         &    &    & etc. &    &    &
    \end{tabular}
  \end{center}
  Today's number appears directly above the number $750$ in the triangle of integers.
}

\def\advent@xxi@abgrid{
  \begin{center}
    \begin{tikzpicture}
      \def\gs{1}
      % Grid
      \foreach \i in {0,...,6}{
          \foreach \j in {0,...,2}{
              \draw[thick] (\i * \gs, \j * \gs) rectangle (\i * \gs + \gs, \j * \gs + \gs);
            }
        }
      % Points
      \fill[color=red] (0, 0) circle (0.2) node[color=red,left] {A};
      \fill[color=blue] (7*\gs, 3*\gs) circle (0.2) node[color=blue,right] {B};
    \end{tikzpicture}
  \end{center}
}
\def\advent@xxi@xii{
  You start at the point marked A in the picture below. You want to get to the point marked B.
  You may travel \textbf{to the right} or \textbf{upwards} along the black lines.

  \advent@xxi@abgrid

  Today's number is the total number of possible routes to get from A to B.
}

\def\advent@xxi@xiii{
  The diagram below shows three circles and two triangles.
  The three circles all meet at one point.
  The vertices of the smaller red triangle are at the centers of the circles.
  The lines connecting the vertices of the larger blue triangle to the point where all three circles meet are diameters of the three circles.

  \begin{center}
    \begin{tikzpicture}[rotate=30,transform shape]
      \def\bcr{3}
      \def\scr{0.55*\bcr}
      \def\sca{34}
      \def\mcr{0.7*\bcr}
      \def\mca{142}
      \def\pr{0.1}

      % Circles
      \draw (0, \bcr) circle (\bcr);
      \draw (\sca: \scr) circle (\scr);
      \draw (\mca: \mcr) circle (\mcr);

      % Points
      \fill (0, 0) circle (\pr);
      \fill (0, \bcr) circle (\pr);
      \fill (0, 2*\bcr) circle (\pr);
      \fill (\sca: \scr) circle (\pr);
      \fill (\sca: 2*\scr) circle (\pr);
      \fill (\mca: \mcr) circle (\pr);
      \fill (\mca: 2*\mcr) circle (\pr);

      % Triangles
      \draw[fill=blue,fill opacity=0.5] (\mca: 2*\mcr) -- (0, 2*\bcr) -- (\sca: 2*\scr) -- cycle;
      \draw[fill=red,fill opacity=0.5] (\mca: \mcr) -- (0, \bcr) -- (\sca: \scr) -- cycle;
    \end{tikzpicture}
  \end{center}

  The area of the smaller red triangle is $226$.
  What is the area of the larger blue triangle?
}

\def\advent@xxi@xiv{
  You start at the point marked A in the picture below.
  You want to get to the point marked B.
  You may travel \textbf{to the right}, \textbf{upwards}, or \textbf{to the left} along the black lines, but you cannot pass along the same line segment more than once.

  \advent@xxi@abgrid

  Today's number is the total number of possible routes to get from A to B.
}

\newcommand\pyramid@advent@xxi@xv[6]{
  \begin{center}
    \begin{tabular}{cccccc}
      (row 1) &    &    & #1   &    &    \\
      (row 2) &    & #2 &      & #3 &    \\
      (row 3) & #4 &    & #5   &    & #6 \\
              &    &    & etc. &    &
    \end{tabular}
  \end{center}
}
\def\advent@xxi@xv{
  The odd numbers are written in a pyramid.

  \pyramid@advent@xxi@xv{1}{3}{5}{7}{9}{11}

  What is the mean of the numbers in the 19th row?
}

\def\advent@xxi@xvi{
  Each clue in this crossnumber is formed of two parts connected by a logical connective: AND means that both parts are true; NAND means that at most one part is true; OR means that at least one part is true; NOR means that neither part is true; XOR means that exactly one part is true; XNOR means that either both parts are false or both parts are true.
  No number starts with $0$.

  \begin{multicols}{2}
    \crossnumstd{}{}{}{}{}{}{}{}{}

    \columnbreak

    \begin{enumerate}
      \item \textbf{1A} is a palindrome XNOR \textbf{1D} is a palindrome.
      \item \textbf{1A} is greater than $350$ NOR \textbf{1D} is less than $150$.
      \item \textbf{3D} is odd NAND \textbf{4A} and \textbf{2D} are equal.
      \item \textbf{3D} is prime XOR \textbf{5A} is odd.
      \item \textbf{4A} is a cube AND \textbf{2D} is a cube.
      \item The sum of the digits of \textbf{3D} is $2$ OR the sum of the digits of \textbf{5A} is $5$.
      \item Today's number is \textbf{1D}.
    \end{enumerate}
  \end{multicols}
}

\def\advent@xxi@xvii{
  The digital product of a number is computed by multiplying together all of its digits. For example, the digital product of $6273$ is $252$.

  Today's number is the smallest number whose digital product is $252$.
}

\def\advent@xxi@xviii{
  Put the digits $1$ to $9$ (using each digit exactly once) in the boxes so that the sums are correct.
  The sums should be read left to right and top to bottom ignoring the usual order of operations.
  For example, $4 + 3 \times 2$ is $14$, not $10$.
  Today's number is the product of the numbers in the red boxes.

  \grid@advent@xxi@xviii{}{}{}{}{}{}{}{}{}
}

\def\advent@xxi@xix{
  The equation $352x^3 - 528x^2 + 90 = 0$ has three distinct real-valued solutions.

  Today's number is the number of integers $a$ such that the equation $352x^3 - 528x^2 + a = 0$ has three distinct real-valued solutions.
}

\def\advent@xxi@xx{
  What is the area of the largest area triangle that has one side of length $32$ and one side of length $19$?
}

\newcommand\grid@advent@xxi@xxi[9]{
  \begin{center}
    \begin{tikzpicture}
      \bigbox{0}{3}{#1}
      \bigbox{1}{3}{#2}
      \bigbox{2}{3}{#3}
      \bbtextr{3}{3}{\textbf{today's number}}

      \bigbox{0}{2}{#4}
      \bigbox{1}{2}{#5}
      \bigbox{2}{2}{#6}
      \bbtextr{3}{2}{prime}

      \bigbox{0}{1}{#7}
      \bigbox{1}{1}{#8}
      \bigbox{2}{1}{#9}
      \bbtextr{3}{1}{square}

      \bbtextb{0}{0}{cube}
      \bbtextb{1}{0}{odd}
      \bbtextb{2}{0}{multiple\\of $11$}
    \end{tikzpicture}
  \end{center}
}
\def\advent@xxi@xxi{
  Arrange the digits $1$–$9$ (using each digit exactly once) so that the three digit number in: the middle row is a prime number; the bottom row is a square number; the left column is a cube number; the middle column is an odd number; the right column is a multiple of $11$.
  The $3$-digit number in the first row is today's number.

  \grid@advent@xxi@xxi{}{}{}{}{}{}{}{}{}
}

\def\advent@xxi@xxii{
  There are $12$ ways of placing $2$ tokens on a $2 \times 4$ grid so that no two tokens are next to each other horizontally, vertically or diagonally:

  \begin{center}
    \begin{tikzpicture}
      % Draw all the grids
      \foreach \gi in {0,1}{
          \foreach \gj in {0,...,5}{
              \foreach \i in {0,1}{
                  \foreach \j in {0,...,3}{
                      \gridbox{5*\gj + \j}{3*\gi + \i}{}
                    }
                }
            }
        }

      % Place token circles
      \gridcirc{0}{4}
      \gridcirc{2}{4}
      \gridcirc{5}{4}
      \gridcirc{8}{4}
      \gridcirc{10}{4}
      \gridcirc{12}{3}
      \gridcirc{15}{4}
      \gridcirc{18}{3}
      \gridcirc{21}{4}
      \gridcirc{23}{4}
      \gridcirc{26}{4}
      \gridcirc{28}{3}
      \gridcirc{0}{0}
      \gridcirc{2}{1}
      \gridcirc{5}{0}
      \gridcirc{8}{1}
      \gridcirc{11}{0}
      \gridcirc{13}{1}
      \gridcirc{15}{0}
      \gridcirc{17}{0}
      \gridcirc{20}{0}
      \gridcirc{23}{0}
      \gridcirc{26}{0}
      \gridcirc{28}{0}
    \end{tikzpicture}
  \end{center}

  Today's number is the number of ways of placing $2$ tokens on a $2 \times 21$ grid so that no two tokens are next to each other horizontally, vertically or diagonally.
}

\def\advent@xxi@xxiii{
  I draw the parabola $y = x^2$ and mark points on the parabola at $x = 17$ and $x = -6$.
  I then draw a straight line connecting these two points.

  At which value of $y$ does this line intercept the $y$-axis?
}

\def\advent@xxi@xxiv{
  The digital product of a number is computed by multiplying together all of its digits.
  For example, the digital product of $1522$ is $20$.

  How many $12$-digit numbers are there whose digital product is $20$?
}

\def\card@xxi@i{
  What is the sum of all the odd integers between $0$ and $30$?
}

\def\card@xxi@ii{
  What is the sum of all the odd integers between $0$ and $5668$?
}

\def\card@xxi@iii{
  What is the smallest integer with a digital sum of $28$ and a digital product of $10000$?
}

\def\card@xxi@iv{
  What is the smallest integer with a digital sum of $41$ and a digital product of $432000$?
}

\def\card@xxi@v{
  What is the area of the largest area dodecagon that will fit inside a circle with area $111185 \pi$?
}

\def\card@xxi@vi{
  What is the area of the largest area heptagon that will fit inside a semicircle with area $115185 \pi$?
}

\def\card@xxi@vii{
  How many terms are there in the (simplified) expansion of $(x + y + z)^2$?
}

\def\card@xxi@viii{
  How many terms are there in the (simplified) expansion of $(x + y + z)^{41172}$?
}

\def\card@xxi@ix{
  What is the largest integer that cannot be written as $4a + 5b$ for non-negative integers $a$ and $b$?
}

\def\card@xxi@x{
  What is the largest integer that cannot be written as $83409a + 66608b$ for non-negative integers $a$ and $b$?
}

\def\card@xxi@xi{
  How many positive integers are there below $100$ whose digits are all non-zero and different?
}

\def\card@xxi@xii{
  How many positive integers are there whose digits are all non-zero and different?
}

\def\card@xxi@xiii{
  What is the only integer for which taking the geometric mean of all its factors (including $1$ and the number itself) gives $2$?
}

\def\card@xxi@xiv{
  What is the only integer for which taking the geometric mean of all its factors (including $1$ and the number itself) gives $25$?
}
